\markboth{第一版前言}{第一版前言} \vspace*{0.0cm}
\thispagestyle{empty}
\vspace*{2.2cm}
\centerline{\zihao{2}\hei{\color{darkblue}{第一版前言}}}\vspace{2cm}

数值计算方法与计算机相结合是本书的特点,也是科学计算发展的需要.
随着计算机的不断发展和进步,优秀的数学软件MATLAB应运而生,
MATLAB一问世就以它强大的功能,被广大科技工作者公认为科学计算最好的软件之一.
为使数值分析与MATLAB更好地结合,我们以最新版MATLAB为平台,编写了新版
《数值计算方法》,这也是数值计算方法教材发展进步的必然结果.

本书介绍了数值计算方法.内容涉及数值计算方法的数学基础、
数值计算方法在工程、科学和数学问题中的应用以及MATLAB程序等,涵盖了经典
数值分析的全部内容:包括非线性方程的数值解法;线性方程组的数值解法;
矩阵特征值与特征向量的数值算法;插值方法;函数逼近;数值积分;数值微分;
常微分方程数值解等.重点讲述数值分析方法的思想和原理,尽可能避免过深的数学理论
和过于繁杂的算法细节.
基于MATLAB是本书的特色.数值计算方法与科学计算软件MATLAB相结合,
有助于读者更有效地利用MATLAB的超强功能,来处理科学计算问题,有助于避免那种学过
数值计算方法但不能上机解决实际问题的现象发生.

在编写过程中,参考了国内已出版的同类教材(参考文献[1]~[23]),吸收了他们的
许多精华和优点,在题材的选取上作了一些变动,适当地增加了一些新内容,对书中
所有的数值方法都给出了MATLAB程序,有大量详实的应用实例可供参考,
有相当数量的习题可供练习.

本书取材新颖、阐述严谨、内容丰富、重点突出、推导详尽、思路清晰、深入浅出、
富有启发性,便于教学与自学.

全书内容由吕同富教授主持编写.具体分工:方秀男编写第1章和第2章;康兆敏
编写第3章;吕同富编写第4章至第9章.吉林大学周蕴时教授,哈尔滨工业大学吴勃英教授,
认真地阅读了本书,纠正了书中很多错误,并提出了许多保贵的修改意见;吉林大学马富明教授审定了书稿.
这里向他们及本书所列参考文献的作者们,清华大学出版社的佟丽霞和王海燕,
以及为本书出版给予热心支持和帮助的朋友们,表示衷心地感谢.

本书可作为理工科本科生研究生数值计算方法课程教材或参考书,也可作为科技人员
使用数值计算方法和MATLAB的参考手册.

出好书,使千百万莘莘学子受益,一直是作者追求的目标.但由于水平所限,尽管作了很大
努力,可能还会有很多不妥甚至是错误,望广大读者给予批评指正,谢谢.



\vspace{2cm}

\hfill 吕同富\hspace{0.2em}

\hfill ltongfu@126.com \hspace{0.2em}

\hfill 2008年03月\hspace{0.2em}
